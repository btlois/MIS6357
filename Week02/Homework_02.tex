\documentclass[12pt,addpoints,answers]{exam}%,answers
\usepackage{hyperref}
\usepackage{graphicx}
\usepackage{amsmath}
\usepackage{amssymb}
\usepackage{enumerate}
\pagestyle{headandfoot}

\newcommand{\code}[1]{\texttt{#1}}

% Go to the following URL for the manual for the exam class:
% http://www-math.mit.edu/~psh/exam/examdoc.pdf

\begin{document}
\firstpageheader{ MIS 6357 }{\bf\large Homework 2}{Due: Saturday, January 28, 2017}
\runningfooter{}{}{}
\begin{center}
%%%%%%%%%%%%%%%%%%%%%%%%%%%%%%%%%%%%%%%%%vvv
%\makebox[\textwidth]{Name:\enspace\hrulefill}
%%%%%%%%%%%%%%%%%%%%%%%%%%%%%%%%%%%%%%%%%^^^
\end{center}



%%%%%%%%%%%%%%%%%%%%%%%%%%%%%%%%%%%%%%%%%vvv



\begin{questions}

\question Write an R function that takes two vectors, $y$ and $\hat{y}$ and returns the RMSE.

\vspace{1cm}

Questions \ref{3.1} : \ref{3.3} are all from the text.  I have provided more specific directions.  These directions supplement and clarify.  They do not replace the questions in the text.

\question \label{3.1} Complete exercise 3.1 in the text.
\begin{parts}
	\part Display two meaningful, interesting histograms.  Create a visualization of the correlation matrix.  The function \code{corrplot} is very useful here.
	\part Use \code{apply} to compute the skewness of each predictor.
	\part Perform at least one Box-Cox transformation and plot 2 histograms: one of the pre-transformed data and one of the post-transformed data.
\end{parts}

\question \label{3.2} Complete exercise 3.2 in the text.
\begin{parts}
	\part Make use of the \code{apply} function.  Also, remove any features with degenerate distributions.
	\part Find the percent of missing data for each of the classes.
	\part Implement the strategy you develop and produce a data set with no missing values.
\end{parts}

\question \label{3.3} Complete exercise 3.3 in the text.
\begin{parts}
	\part Just do what it says.
	\part Also, remove any features with degenerate distributions.
	\part Perform PCA on the data and decide how many principal components to keep.
\end{parts}











\end{questions}
%%%%%%%%%%%%%%%%%%%%%%%%%%%%%%%%%%%%%%%%%^^^

\end{document} 