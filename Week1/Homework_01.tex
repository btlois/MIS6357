\documentclass[12pt,addpoints,answers]{exam}%,answers
\usepackage{hyperref}
\usepackage{graphicx}
\usepackage{amsmath}
\usepackage{amssymb}
\usepackage{enumerate}
\pagestyle{headandfoot}

\newcommand{\code}[1]{\texttt{#1}}

% Go to the following URL for the manual for the exam class:
% http://www-math.mit.edu/~psh/exam/examdoc.pdf

\begin{document}
\firstpageheader{ MIS 6357 }{\bf\large Homework 1}{Due: Friday, January 20, 2016}
\runningfooter{}{}{}
\begin{center}
%%%%%%%%%%%%%%%%%%%%%%%%%%%%%%%%%%%%%%%%%vvv
%\makebox[\textwidth]{Name:\enspace\hrulefill}
%%%%%%%%%%%%%%%%%%%%%%%%%%%%%%%%%%%%%%%%%^^^
\end{center}



%%%%%%%%%%%%%%%%%%%%%%%%%%%%%%%%%%%%%%%%%vvv



\begin{questions}

\question Functions and loops.
\begin{parts}
	\part \label{is_prime} Write a function to test whether an integer \code{a} is prime. Recall (or be surprised to learn!) that 1 is not a prime.
	\part \label{n} Using your function from \eqref{is_prime} write a second function that takes an input \code{n} and returns the $n^{\mathrm{th}}$ prime number.
	\part Use the \code{sapply} function and your function from \eqref{n} to print the first 20 prime numbers.
\end{parts}

\question Consider the following vector:

\code{x <- c(91, NA, 90, 7, 67, NA, 87, 36, 2, 93, 27, 16)}.

Using only one line of code each, perform the following operations:
\begin{parts}
	\part Remove the \code{NA} values from \code{x}.
	\part Print the first, third, and eighth elements of \code{x}.
	\part Print the elements of \code{x} that are greater than 50.
	\part Print the odd elements of \code{x}.
\end{parts}

\question The \code{mtcars} dataset is included with R.  Use \code{?mtcars} to learn about the data.
\begin{parts}
	\part Use \code{ggplot} to make a scatter plot of fuel efficiency and engine size.
	\part Add a horizontal line to the plot at the median of the y values.
	\part Add a vertical line to the plot at the median of the x values.
	\part Make a box plot of $\frac{1}{4}$ mile time vs number of engine cylinders.
		Describe what each aspect of your plot represents.  i.e. What do the lines, box widths, whiskers, etc. represent?
\end{parts}

\question Recall the trade-off between prediction and interpretation.
\begin{parts}
	\part \label{pred} Describe a situation where prediction would be more important than interpretation.
	\part Describe a second situation where interpretation is relatively more important than in your answer to (\ref{pred}).
\end{parts}

















\end{questions}
%%%%%%%%%%%%%%%%%%%%%%%%%%%%%%%%%%%%%%%%%^^^

\end{document} 